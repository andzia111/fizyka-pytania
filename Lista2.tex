\subsection{Na czym polega katastrofa ultrafioletowa?}

odpowiedź

\subsection{Jakie są eksperymentalne dowody tego, że światło istnieje, się emituje oraz jest absorbowane porcjami (kwantami)?}

odpowiedź

\subsection{Przedyskutuj zjawisko interferencji fal}

odpowiedź

\subsection{Czym są pakiety falowe? Problem normalizacji, przekształcenia pomiędzy przestrzenią położenia oraz przestrzenią pędu.}

odpowiedź

\subsection{Stany kwantowe, operatory oraz równanie Schrödingera w interpretacji Feynmana}
(z różnych źródeł, fiszek fishcards, chatu, notatek)
\textbf{Stan kwantowy} to matematyczny opis systemu kwantowego, który zawiera wszystkie informacje o jego właściwościach.  
\textit{Podstawowe elementy stanu kwantowego:}
\begin{itemize}
  \item \textit{Funkcja falowa} \(\Psi(\mathbf{r},t)\) lub ogólniej wektor stanu \(\lvert \psi \rangle\) w przestrzeni Hilberta.
  \item \textit{Przestrzeń Hilberta} \(\mathcal{H}\) – zespolona przestrzeń wektorowa z iloczynem skalarnym, w której żyją wektory stanu.
\textit{Rodzaje stanów:}
\begin{itemize}
  \item{Stan czysty} 
  \item{Stan mieszany} 
  \textit{Superpozycja} to możliwość istnienia jednocześnie w wielu stanach bazowych 
  
\textbf{Operatory}
Operator to matematyczny obiekt, który działa na funkcje falowe i odpowiada za pomiar wielkości fizycznych.
Amplitudę że przejdziemy ze stanu $+S$ do $0R$ możemy zapisać jako
\begin{equation*}
	a = \langle\ 0R \vert \dots \rangle \langle\ \dots \vert \dots \rangle \dots \langle\ \dots \vert \dots \rangle \langle\ \dots \vert +S \rangle  = \langle\ 0R \vert A \vert +S \rangle 
\end{equation*}
gdzie $A$ to są różne operacje związane ze zbiorem urządzeń. Przez $\hat{A}$ będziemy oznaczać operator generalny.
Możemy też zapisać macierz operatora $\hat{A}$. Niech $\vert +S \rangle$, $\vert 0S \rangle$, $\vert -S \rangle$ będą stanami naszej bazy. Wtedy mamy
\begin{table}[ht]
	\centering
	\begin{tabular}[t]{c|c|c|c}
		& $+$ & $0$ & $-$\\
		\hline
		$+$ & $\langle\ + \vert \hat{A} \vert + \rangle $ & $\langle\ + \vert \hat{A} \vert 0 \rangle $ & $\langle\ + \vert \hat{A} \vert - \rangle $ \\
		\hline
		$0$ & $\langle\ 0 \vert \hat{A} \vert + \rangle $ & $\langle\ 0 \vert \hat{A} \vert 0 \rangle $ & $\langle\ 0 \vert \hat{A} \vert - \rangle $ \\
		\hline
		$-$ & $\langle\ - \vert \hat{A} \vert + \rangle $ & $\langle\ - \vert \hat{A} \vert 0 \rangle $ & $\langle\ - \vert \hat{A} \vert - \rangle $ \\
	\end{tabular}
	\label{tab:macierzoperatora}
\end{table}

Powyższe nazywamy macierzą operatora $\hat{A}$ w bazie takiej jak wyżej podanej. Dajemy też czapkę nad operatorem aby zawsze było wiadomo że chodzi nam o operator, ale nie zawsze trzeba to pisać. Operator nigdy nie jest zależny od bazy ale macierz operatora zawsze jest w jakiejś bazie.
\textit{Najważniejsze operatory w mechanice kwantowej}

\begin{enumerate}
  \item \textbf{Operator Hamiltona} \(\hat{H}\) \\
    Opisuje całkowitą energię układu i steruje jego ewolucją czasową.

  \item \textbf{Operator położenia} \(\hat{x}\) \\
    Przypisuje wartość położenia cząstki.

  \item \textbf{Operator pędu} \(\hat{p}\) \\
    Odpowiada wartości pędu, w reprezentacji położenia jest operatorem różniczkowym.

  \item \textbf{Operatory spinowe} \(\hat{S}_i\) (np. \(\hat{S}_x, \hat{S}_y, \hat{S}_z\)) \\
    Opisują własności spinowe cząstek, np. spinu \(1/2\).

  \item \textbf{Operator projekcji} \(\hat{P}\) \\
    Rzutuje stan na podprzestrzeń odpowiadającą określonej wartości pomiaru.

  \item \textbf{Operator unitarny} \(\hat{U}\) \\
    Opisuje ewolucję jednostkową i zmiany baz wektorów stanu.

  \item \textbf{Operator gęstości} \(\rho\) \\
    Opisuje stan mieszany i statystyczny układ kwantowy.
\end{enumerate}
\textbf{ równanie Schrödingera w interpretacji Feynmana}
nie wiem o co tu chodzi w notatkach jest jedynie coś nazwane ,,Eksperyment S-G w interpretacji Feynmana". Wkleje to ponizej ale nie wiem czy o to chodzi. 
Eksperyment z dwoma szczelinami.
Prawdopodobieństwo, że elektron ze stanu S przejdzie do stanu X \newline [Stan $X$] $\leftarrow$ [Stan $S$] = $\vert\langle X\vert S\rangle\vert^2$ będziemy nazywać amplitudą.
Amplitudę ze stanu $S$ do stanu $X$ możemy w tym przypadku zapisać dwojako, w postaci $\langle X\vert1\rangle, \langle 1\vert S\rangle$ jako przejścia ze stanu $S$ do stanu $1$ a następnie ze stanu $1$ do stanu $X$, oraz analogicznie w postaci $\langle X\vert2\rangle , \langle 2\vert S\rangle$. Przepiszemy korzystając z tych oznaczeń zasadę superpozycji. Jeżeli chcemy przejść ze stanu $S$ do stanu $X$ to będziemy to zapisywać jako sumę amplitud $\langle X\vert S\rangle = \langle X\vert1\rangle\langle 1\vert S\rangle + \langle X\vert2\rangle \langle 2\vert S\rangle$. Jeżeli chcemy zapisać prawdopodobieństwo to musimy obliczyć z tego moduł do kwadratu i obliczyć to wszystko w liczbach zespolonych $\vert\langle X\vert S\rangle \vert^2 = \vert[...]\vert^2$. 
\textit{Pytanie} co oznacza superpozycja? Jeżeli chcemy obliczyć amplitudę to jest to suma poszczególnych amplitud. Amplituda $S$ do $Z$ jest superpozycją przejścia przez szczeliny.
\subsection{Twierdzenie Ehrenfesta}
(notatki str. 27)
$$
\frac{d}{dt} \langle x \rangle = \frac{1}{m} \int \psi^* \left(-i\hbar \frac{\partial}{\partial x}\right) \psi \, d\vec{r} = \frac{1}{m} \langle \hat{p}_x \rangle.
$$

$$
\frac{d}{dt} \langle \hat{p}_x \rangle = - \left\langle \frac{\partial V}{\partial x} \right\rangle.
$$

Powyższe dwa równania stanowią treść twierdzenia Ehrenfesta i pokazują, że średnie wartości położenia i pędu w mechanice kwantowej zmieniają się zgodnie z klasycznymi równaniami ruchu. 


\subsection{Obserwable, równanie Schrödingera, zależne oraz niezależne od czasu, własne stany, własne energie}

\textbf{Obserwable:}
W mechanice kwantowej \textbf{obserwabla} (ang. observable) to fizyczna wielkość, którą można zmierzyć eksperymentalnie, np. położenie, pęd, energia czy spin. Każdej obserwabli odpowiada hermitowski operator $\hat{A}$ działający na funkcje falowe przestrzeni Hilberta. Wartość średnia obserwabli $\hat{A}$ w stanie opisanym funkcją falową $\psi(\vec{r}, t)$ dana jest przez wyrażenie:
$$
\langle \hat{A} \rangle = \int \psi^*(\vec{r}, t) \hat{A} \psi(\vec{r}, t) \, d\vec{r}.
$$

Operator $\hat{A}$ musi być hermitowski, aby wartości średnie $\langle \hat{A} \rangle$ były liczbami rzeczywistymi, zgodnie z wymaganiami eksperymentu:
$$
\langle \hat{A} \rangle \in \mathbb{R}.
$$

Dla hermitowskiego operatora $\hat{A}$ zachodzi również
$$
\langle \hat{A} \rangle = \int \psi^* (\hat{A} \psi) \, d\vec{r} = \int (\hat{A} \psi)^* \psi \, d\vec{r}.
$$

\textbf{Przykłady obserwabli:}

\begin{itemize}
    \item Średnia wartość położenia:
    $$
    \langle \hat{\vec{r}} \rangle = \int \psi^*(\vec{r}, t) \vec{r} \psi(\vec{r}, t) \, d\vec{r}.
    $$

    \item Średnia wartość funkcji $f(\vec{r}, t)$:
    $$
    \langle f(\vec{r}, t) \rangle = \int \psi^*(\vec{r}, t) f(\vec{r}, t) \psi(\vec{r}, t) \, d\vec{r}.
    $$

    \item Średnia wartość pędu (w reprezentacji położeniowej, po transformacji Fouriera):
    $$
    \langle \vec{p} \rangle = \int \psi^*(\vec{r}, t) \left( -i\hbar \vec{\nabla} \right) \psi(\vec{r}, t) \, d\vec{r}.
    $$
\end{itemize}
\textbf{Równanie Schorninberga zależne od czasu:}

\begin{equation}
  i \hbar \frac{\partial}{\partial t} \Psi(\mathbf{r},t)
  = \hat{H}\,\Psi(\mathbf{r},t),
\end{equation}

gdzie hamiltonian dla jednej cząstki w potencjale \(V(\mathbf{r},t)\) ma postać:
\begin{equation}
  \hat{H}
  = -\frac{\hbar^2}{2m} \,\nabla^2 \;+\; V(\mathbf{r},t).
\end{equation}

\bigskip
Poniżej zestawienie znaczenia poszczególnych symboli:

\begin{itemize}
  \item[\(\Psi(\mathbf{r},t)\)] funkcja falowa cząstki, zależna od współrzędnych przestrzennych \(\mathbf{r}\) i czasu \(t\).
  \item[\(i\)] jednostka urojona, spełniająca \(i^2 = -1\).
  \item[\(\hbar\)] stała Plancka podzielona przez \(2\pi\), \(\hbar = \frac{h}{2\pi}\).
  \item[\(\frac{\partial}{\partial t}\)] pochodna cząstkowa względem czasu \(t\).
  \item[\(\hat{H}\)] operator Hamiltona, opisujący energię całkowitą układu.
  \item[\(m\)] masa cząstki.
  \item[\(\nabla^2\)] operator Laplace’a
  \item[\(V(\mathbf{r},t)\)] potencjał, w którym porusza się cząstka; może zależeć od położenia \(\mathbf{r}\) i czasu \(t\).
\end{itemize}

\textbf{Równanie Schorninberga niezależne od czasu:}
\[
\hat{H}\,\psi(\mathbf{r}) = E\,\psi(\mathbf{r}),
\]
gdzie
\[
\hat{H} = -\frac{\hbar^2}{2m}\,\nabla^2 + V(\mathbf{r}).
\]

\bigskip
\noindent
Znaczenie symboli:
\begin{itemize}
  \item[\(\hat{H}\)] operator Hamiltona (energia całkowita układu),
  \item[\(\psi(\mathbf{r})\)] funkcja własna (stacjonarna funkcja falowa),
  \item[\(E\)] wartość własna (energia stanu),
  \item[\(\hbar\)] zredukowana stała Plancka (\(\hbar = h/(2\pi)\)),
  \item[\(\nabla^2\)] operator Laplace’a 
    \(\bigl(\nabla^2 = \tfrac{\partial^2}{\partial x^2}
                        + \tfrac{\partial^2}{\partial y^2}
                        + \tfrac{\partial^2}{\partial z^2}\bigr)\),
  \item[\(V(\mathbf{r})\)] potencjał zależny od położenia \(\mathbf{r}\).
\end{itemize}
\textbf{własności własne egergii własne)
Funkcje własne operatora Hamiltona $\hat{H}$, oznaczane jako $\psi_E(\vec{r})$, spełniają równanie Schrödingera $\hat{H} \psi_E = E \psi_E,$ gdzie $E$ jest odpowiadającą im wartością własną, interpretowaną jako energia stanu. Zakładamy, że funkcje $\psi_E(\vec{r})$ są unormowane
$$
\int \psi_E^*(\vec{r}) \psi_E(\vec{r}) \, d\vec{r} = 1,
$$
oraz ortogonalne względem siebie dla różnych wartości energii
$$
\int \psi_E^*(\vec{r}) \psi_{E'}(\vec{r}) \, d\vec{r} = 0
\quad \text{dla } E \ne E'.
$$
Dla $E \ne E'$ całka musi być równa zeru, czyli funkcje własne odpowiadające różnym wartościom energii są ortogonalne.

Z faktu, że operator Hamiltona jest hermitowski, wynika również, że jego funkcje własne tworzą pełną bazę przestrzeni Hilberta. Oznacza to, że każdą funkcję falową $\Psi(\vec{r}, t)$ można przedstawić jako kombinację liniową funkcji własnych Hamiltonianu
$$
\Psi(\vec{r}, t) = \sum_E C_E(t) \psi_E(\vec{r}),
$$
gdzie suma biegnie po wszystkich stanach odpowiadających różnym energiom, w tym zdegenerowanym.

\subsection{Proste zagadnienia: swobodna cząstka}

Rozwiązanie równania Schrödingera dla swobodnej cząstki to fale płaskie o postaci $\Psi(x) = A e^{ikx} + B e^{-ikx}$, gdzie
$k$ jest wektorem falowym związanym z energią kinetyczną cząstki. Energia jest zawsze nieujemna, a funkcje falowe muszą być
ograniczone, co wymusza rzeczywiste wartości $k$.


\begin{equation*}
    - \frac{\hbar^2}{2m} \frac{\partial^2}{\partial x^2} \psi(x) = E \psi(x)
\end{equation*}

gdzie:
\begin{equation*}
    k = \frac{\sqrt{2mE}}{\hbar}
\end{equation*}
%
\begin{equation*}
    \psi(x) = A e^{ikx} + B e^{-ikx}
\end{equation*}

$ k \in \mathbb{R} $, bo inaczej mamy rozbieżne $\psi(x)$. \\
$ E >= 0 $, bo $ k = \frac{\sqrt{2mE}}{\hbar^2} $, a $ 2m > 0 $, $ \hbar > 0 $.

Natomiast $E$ wyraża się wzorem:
\begin{equation*}
    E = \frac{\hbar^2 k^2}{2m}
\end{equation*}


\subsection{Proste zagadnienia: potencjał w kształcie schodków, bariery, studni kwadratowej}

odpowiedź

\subsection{Proste zagadnienia: oscylator harmoniczny}

odpowiedź

\subsection{Formalizm mechaniki kwantowej, postulaty}

odpowiedź

\subsection{Klasy i własności operatorów. Komutatory}

W mechanice kwantowej operatory reprezentują obserwowalne wielkości fizyczne, takie jak pozycja czy pęd. Operatory działają na przestrzeni stanów (np. w przestrzeni Hilberta) i mogą mieć różne własności: być hermitowskie (samosprzężone), jednostkowe czy projekcyjne. Hermitowskie operatory odpowiadają mierzalnym wartościom rzeczywistym.

Klasa operatorów określa ich charakter (np. ograniczone, nieograniczone). Ważnym pojęciem są komutatory operatorów
\[
[A, B] = AB - BA.
\]
Jeśli
\[
[A, B] = 0,
\]
to operatory się komutują, co oznacza, że można jednocześnie mierzyć odpowiadające im wielkości z pełną precyzją. Niezerowy komutator wskazuje na fundamentalne ograniczenia pomiarowe, jak w przypadku zasady nieoznaczoności Heisenberga.

\subsection{Zasada nieoznaczoności Heisenberga}

Zasada nieoznaczoności Heisenberga wyraża fundamentalne ograniczenie precyzji, z jaką można jednocześnie znać wartości pewnych par wielkości fizycznych, np. położenia \( \hat{x} \) i pędu \( \hat{p} \). Formalnie wyraża się to nierównością

\[
\Delta x \, \Delta p \geq \frac{\hbar}{2},
\]

gdzie \( \Delta x \) i \( \Delta p \) to odchylenia standardowe pomiarów operatorów położenia i pędu, a \( \hbar \) to zredukowana stała Plancka.

Ta zasada wynika z faktu, że operatory położenia i pędu nie komutują, tzn.

\[
[\hat{x}, \hat{p}] = i \hbar,
\]

co implikuje, że nie istnieje wspólny zbiór własnych wektorów obu operatorów, a więc nie można jednocześnie przypisać im dokładnych wartości.

\subsection{Operator momentu pędu, uogólniony operator momentu pędu, operator spinu: własne funkcje i wartości}


Operator momentu pędu \(\hat{\mathbf{L}} = (\hat{L}_x, \hat{L}_y, \hat{L}_z)\) opisuje moment pędu orbitalnego cząstki. Składowe operatora spełniają następujące relacje komutacyjne:

\[
[\hat{L}_i, \hat{L}_j] = i \hbar \epsilon_{ijk} \hat{L}_k,
\]

gdzie \(\epsilon_{ijk}\) to symbol Levi-Civity, a \(i, j, k \in \{x,y,z\}\).

Operator kwadrat momentu pędu \(\hat{L}^2 = \hat{L}_x^2 + \hat{L}_y^2 + \hat{L}_z^2\) oraz składowa \(\hat{L}_z\) mają wspólny układ własnych funkcji \(|l, m\rangle\), dla których zachodzą własności:

\[
\hat{L}^2 |l, m\rangle = \hbar^2 l(l+1) |l, m\rangle, \quad \hat{L}_z |l, m\rangle = \hbar m |l, m\rangle,
\]

gdzie \(l = 0, 1, 2, \ldots\), a \(m = -l, -l+1, \ldots, l\).

Uogólniony operator momentu pędu \(\hat{\mathbf{J}} = \hat{\mathbf{L}} + \hat{\mathbf{S}}\) łączy moment pędu orbitalny \(\hat{\mathbf{L}}\) oraz spin \(\hat{\mathbf{S}}\).

Operator spinu \(\hat{\mathbf{S}}\) opisuje wewnętrzny moment pędu cząstek, niezwiązany z ruchem orbitalnym. Składowe spinu również spełniają relacje komutacyjne analogiczne do momentu pędu:

\[
[\hat{S}_i, \hat{S}_j] = i \hbar \epsilon_{ijk} \hat{S}_k.
\]

Dla spinu \(s\) (np. \(s = \frac{1}{2}\) dla elektronu), własne wartości operatorów \(\hat{S}^2\) i \(\hat{S}_z\) są:

\[
\hat{S}^2 |s, m_s\rangle = \hbar^2 s(s+1) |s, m_s\rangle, \quad \hat{S}_z |s, m_s\rangle = \hbar m_s |s, m_s\rangle,
\]

gdzie \(m_s = -s, -s+1, \ldots, s\).

Własne funkcje momentu pędu i spinu tworzą bazę przestrzeni stanów kwantowych, na której można opisywać stan cząstki z uwzględnieniem zarówno ruchu orbitalnego, jak i spinu.

\subsection{Atom wodoru: stany własne oraz energie własne}


Atom wodoru opisuje równanie Schrödingera z potencjałem Coulomba:

\[
\hat{H} = -\frac{\hbar^2}{2m} \nabla^2 - \frac{e^2}{4 \pi \varepsilon_0 r},
\]

gdzie \(m\) to masa elektronu, \(e\) ładunek elementarny, a \(r\) odległość elektronu od jądra.

Stany własne \(|n, l, m\rangle\) są jednocześnie własnymi funkcjami operatorów:

\[
\hat{H}|n, l, m\rangle = E_n |n, l, m\rangle,
\]

\[
\hat{L}^2 |n, l, m\rangle = \hbar^2 l(l+1) |n, l, m\rangle,
\]

\[
\hat{L}_z |n, l, m\rangle = \hbar m |n, l, m\rangle,
\]

gdzie liczby kwantowe przyjmują wartości

\[
n = 1, 2, 3, \ldots, \quad l = 0, 1, \ldots, n-1, \quad m = -l, -l+1, \ldots, l.
\]

Energia własna jest określona wzorem:

\[
E_n = - \frac{m e^4}{2 (4 \pi \varepsilon_0)^2 \hbar^2} \frac{1}{n^2} = - \frac{13.6\,\mathrm{eV}}{n^2}.
\]

Energia zależy wyłącznie od głównej liczby kwantowej \(n\), co prowadzi do degeneracji poziomów energetycznych względem liczb \(l\) i \(m\).

\subsection{Własności funkcji falowych bozonów oraz fermionów}


W mechanice kwantowej funkcje falowe bozonów i fermionów różnią się symetrią względem zamiany dwóch identycznych cząstek:

\begin{itemize}
    \item \textbf{Bozony} mają symetryczne funkcje falowe, tzn. przy zamianie cząstek
    \[
    \Psi(\ldots, \mathbf{r}_i, \ldots, \mathbf{r}_j, \ldots) = + \Psi(\ldots, \mathbf{r}_j, \ldots, \mathbf{r}_i, \ldots).
    \]
    
    \item \textbf{Fermiony} mają antysymetryczne funkcje falowe, tzn.
    \[
    \Psi(\ldots, \mathbf{r}_i, \ldots, \mathbf{r}_j, \ldots) = - \Psi(\ldots, \mathbf{r}_j, \ldots, \mathbf{r}_i, \ldots).
    \]
\end{itemize}

Antysymetria funkcji falowej fermionów prowadzi do zasady Pauliego wykluczania, która zabrania zajmowania tego samego stanu kwantowego przez dwie identyczne fermiony.

Symetria funkcji falowej bozonów pozwala na zajmowanie tego samego stanu kwantowego przez wiele cząstek, co jest podstawą efektów takich jak kondensacja Bosego-Einsteina.
